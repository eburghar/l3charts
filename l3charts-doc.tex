\documentclass{ltxdoc}
\usepackage{l3charts}
\usepackage{minted}
\usepackage[parfill]{parskip}
\usepackage{geometry}
\usepackage{hyperref}
\usepackage{booktabs}

% Change the layout
%% colored links
\hypersetup{
	colorlinks,
	citecolor=black,
	filecolor=black,
	linkcolor=red,
	urlcolor=blue
}
%% smaller margins
\geometry{
	a4paper,
	total={150mm,257mm},
	left=40mm,
	top=20mm,
}
%% remove first paragraph line indentation
\setlength{\parindent}{0pt}


% Add macro and env to avoid repetitions
%% format label white on black
\NewDocumentCommand\textinv{m}{\colorbox{black}{\textcolor{white}{#1}}}
%% format environment name
\NewDocumentCommand\env{m}{\texttt{#1} (\emph{env})}
%% format environement use
\NewDocumentCommand\envdef{moO{...}}{\cs{begin}\{\texttt{#1}\}#2\\\indent#3\\\cs{end}\{\texttt{#1}\}}
%% format environment definition
\NewDocumentEnvironment{Environment}{moo}{%%
	\begin{environment}{#1}
		\envdef{#1}[#2][#3]\\\\
		} {%%
	\end{environment}
}
%% format property list for options
\NewDocumentEnvironment{props}{o}{%%
	\NewDocumentCommand\prop{mom}{\texttt{##1} & \IfValueT{##2}{\texttt{##2}} & ##3\\ \addlinespace}
	\begin{tabular}{lp{0.3\linewidth}p{0.4\linewidth}} \toprule
		\textsf{Key} & \textsf{Default value} & \textsf{Description} \\
		\midrule
		} {
    \bottomrule
	\end{tabular}
}
%% typeset TikZ
\NewDocumentCommand{\TikZ}{}{Ti\textit{k}Z\xspace}

% Start document

\title{The \texttt{l3charts} package}
\author{Éric BURGHARD}
\date{2022/07/01}
\changes{0.1.0}{2022/07/01}{initial version}

\RecordChanges
\PageIndex

\begin{document}
\maketitle
\begin{abstract}
	This package defines a few simple TikZ{} charts that can be drawn using \LaTeX{}
	environments. This has mainly been developped as an experimentation of \texttt{expl3}
	for checking what \LaTeX3 really brought to facilitate package developpement (expansion
	control, clist, seq, prop, ...).
\end{abstract}
\tableofcontents

\section{About this documentation}

In my opinion \LaTeX{} literate programming is just an ugly hack that
turns the code and the documentation unreadable. If you used modern tools like
\href{https://doc.rust-lang.org/cargo/commands/cargo-doc.html}{\texttt{cargo doc}} which typeset
index, and crossreference your code documentation without the need of inserting any command, you
probably also feel that \texttt{docstrip} is perhaps the component of \LaTeX{} which aged the most.

Perhaps the naming convention of \LaTeX3 would one day allow to have more powerful
tools for automatic documentation extraction, but in the meantime, I think that writing the
doc separately is easier and more maintenable.

\section{kiviat chart}

The \href{https://en.wikipedia.org/wiki/Radar_chart}{kiviat chart} or \emph{radar chart} allows
to represent one or several set along several dimensions.

\DescribeEnv{kiviatchart}
\begin{Environment}{kiviatchart}[\oarg{clist}]
	Environment that hold a kiviat chart. \oarg{clist} is a list of the following options

	\begin{props}[\oarg{clist}]
		\prop{radius}[3.5cm]{maximal diagram radius}
		\prop{labels-radius}[3.5cm]{radius to put dimension labels}
		\prop{units}[5]{number of scale units}
		\prop{*}{all other options are passed to \env{tikzpicture}}
	\end{props}

	A \env{kiviatchart} should begin with a \env{dims} followed by one or several \env{set}.
\end{Environment}

\subsection{Dimensions}

\begin{Environment}{dims}[\oarg{clist}][\cs{value}\oarg{clist}\marg{label}\\\indent...]
	\oarg{clist} is a list of the following options

	\begin{props}[\oarg{clist}]
		\prop{dim-options}[\{opacity=0.8\}]{TikZ{} options for drawing dimensions axis}
		\prop{unit-options}[\{opacity=0.3\}]{TikZ{} options for drawing unit polygones}
		\prop{label-options}[\{opacity=0.5, below\}]{TikZ{} options drawing for unit labels}
		\prop{label-cs}[identity]{name of the cs used to format labels}
		\prop{unit-cs}[tinyt]{name of the cs used to format labels}
	\end{props}

	\begin{macro}{tinyt}
		Macro used to format unit labels
		\begin{minted}{latex}
\cs_new:Npn \tinytt #1 {\tiny\texttt{#1}}
		\end{minted}
	\end{macro}

	\begin{macro}{value}
		Inside \env{dims}, \cs{value}\oarg{clist}\marg{label} is used to add a dimension
		to the kiviat chart
	\end{macro}
\end{Environment}

\subsection{Sets}

\begin{Environment}{set}[\oarg{clist}][\cs{value}\marg{int}\\\indent...]
	\env{set} is used to add a new set to the kiviat chart. \oarg{clist} is a list of the
	following options :

	\begin{props}[\oarg{clist}]
		\prop{dot-options}[\{fill, circle, inner~sep=1pt\}]{options for polygone node}
		\prop{*}[\{color=black, line~width=1.5pt, opacity=1, fill~opacity=0.3,
		fill=gray\}]{all other options are passed to \cs{draw} cs which draws the polygone}
	\end{props}

	\begin{macro}{value}
		Inside \env{set}, \cs{value}\marg{int} is used to add a value to the set. There must be the same number
		of \cs{value} inside \env{set} and \env{dims}, and each \cs{value} corresponds to the dimension in \env{dims}
		at the same index.
	\end{macro}
\end{Environment}

\subsection{Examples}

\subsubsection{Simple}

Use \texttt{label-cs} to call a \cs{textinv} to format the labels.

\begin{macro}{textinv}
\begin{minted}{latex}
\NewDocumentCommand\textinv{m}{\colorbox{black}{\textcolor{white}{#1}}}
\end{minted}
\end{macro}

\begin{minted}[linenos,fontsize=\footnotesize]{latex}
\begin{kiviatchart}[scale=0.75]
  \begin{dims}[label-cs=textinv]
    \value[above]{Dim 1}
    \value[above]{Dim 2}
    \value[left]{Dim 3}
    \value[right]{Dim 4}
    \value[right]{Dim 5}
  \end{dims}
  \begin{set}
    \value{4}
    \value{3}
    \value{4}
    \value{5}
    \value{3}
  \end{set}
\end{kiviatchart}
\end{minted}

\begin{kiviatchart}[scale=0.75]
  \begin{dims}[label-cs=textinv]
    \value[above]{Dim 1}
    \value[above]{Dim 2}
    \value[left]{Dim 3}
    \value[right]{Dim 4}
    \value[right]{Dim 5}
  \end{dims}
  \begin{set}
    \value{4}
    \value{3}
    \value{4}
    \value{5}
    \value{3}
  \end{set}
\end{kiviatchart}

\subsubsection{Multi-set}

Each set set its own color and point shape.

\begin{minted}[linenos,fontsize=\footnotesize]{latex}
\begin{kiviatchart}[scale=0.75]
  \begin{dims}
    \value[above]{Dim 1}
    \value[above]{Dim 2}
    \value[left]{Dim 3}
    \value[right]{Dim 4}
    \value[right]{Dim 5}
  \end{dims}
  \begin{set}[fill=red,dot-options={fill,rectangle,inner sep=2pt}]
    \value{4}
    \value{3}
    \value{4}
    \value{5}
    \value{2}
  \end{set}
  \begin{set}[fill=green,dot-options={fill,diamond,inner sep=2pt}]
    \value{2}
    \value{4}
    \value{3}
    \value{4}
    \value{5}
  \end{set}
  \begin{set}[fill=blue,dot-options={fill,semicircle,inner sep=2pt}]
    \value{5}
    \value{2}
    \value{4}
    \value{3}
    \value{4}
  \end{set}
\end{kiviatchart}
\end{minted}

\begin{kiviatchart}[scale=0.75]
	\begin{dims}
		\value[above]{Dim 1}
		\value[above]{Dim 2}
		\value[left]{Dim 3}
		\value[right]{Dim 4}
		\value[right]{Dim 5}
	\end{dims}
	\begin{set}[fill=red,dot-options={fill,rectangle,inner sep=2pt}]
		\value{4}
		\value{3}
		\value{4}
		\value{5}
		\value{2}
	\end{set}
	\begin{set}[fill=green,dot-options={fill,diamond,inner sep=2pt}]
		\value{2}
		\value{4}
		\value{3}
		\value{4}
		\value{5}
	\end{set}
	\begin{set}[fill=blue,dot-options={fill,semicircle,inner sep=2pt}]
		\value{5}
		\value{2}
		\value{4}
		\value{3}
		\value{4}
	\end{set}
\end{kiviatchart}

\subsection{Todo}

At the moment the environments are not user friendly. We could provide basic sanity checks,
with error messages when they failed.

\begin{itemize}
	\item one and only one \env{dims} declared before any \env{set}
	\item \env{set} has the same number of \cs{value} than \env{dims}
	\item \cs{value} in \env{set} is between 0 and \texttt{units}
\end{itemize}

\section{ball chart}

\subsection{Definition}

\begin{Environment}{ballchart}[\oarg{clist}]
	Environment that hold a ball chart. \oarg{clist} is a list of the following keyval

	\begin{props}[\oarg{clist}]
		\prop{n}[]{the number of circles (\textbf{required})}
		\prop{v-sep}[0.1]{vertical separator}
		\prop{h-sep}[0.5]{horizontal separator (circle)}
		\prop{radius}[0.25]{radius}
		\prop{gap}[0.05]{gap between circle}
		\prop{label-cs}[identity]{cs name to format labels}
		\prop{fill-options}[\{fill=black\}]{options to fill balls with}
		\prop{draw-options}[\{draw=black!30\}]{options to draw balls with}
		\prop{label-options}[\{left\}]{options for dimensions axis}
		\prop{*}{all other options are passed to \env{tikzpicture}}
	\end{props}

	\begin{macro}{value}
		Inside \env{ballchart}, \cs{value}\marg{label}\marg{percent} is used to add a new bar.
	\end{macro}

\end{Environment}

\subsection{Examples}

\subsubsection{Simple}

\begin{minted}[linenos,fontsize=\footnotesize]{latex}
\begin{ballchart}[n=5, draw-options={draw=none}]
  \value{Dim 1}{95}
  \value{Dim 2}{80}
  \value{Dim 3}{80}
  \value{Dim 4}{20}
\end{ballchart}
\end{minted}

\begin{ballchart}[n=5, draw-options={draw=none}]
  \value{Dim 1}{95}
  \value{Dim 2}{80}
  \value{Dim 3}{80}
  \value{Dim 4}{20}
\end{ballchart}

\subsubsection{Caped}

Format labels, show circles, change color, add more circles.

\begin{minted}[linenos,fontsize=\footnotesize]{latex}
\begin{ballchart}[n=5, draw-options={draw=black,dotted}]
  \value{Dim 1}{95}
  \value{Dim 2}{80}
  \value{Dim 3}{80}
  \value{Dim 4}{20}
\end{ballchart}
\end{minted}

\begin{ballchart}[n=6, label-cs=textinv, v-sep=0.2, fill-options={fill=red!50}, draw-options={draw=black,dotted}]
  \value{Dim 1}{95}
  \value{Dim 2}{80}
  \value{Dim 3}{80}
  \value{Dim 4}{20}
\end{ballchart}

\section{bar chart}

\subsection{Definition}

\begin{Environment}{barchart}[\oarg{clist}]
	Environment that hold a bar chart. \oarg{clist} is a list of the following keyval

	\begin{props}[\oarg{clist}]
		\prop{width}{maximum width (\textbf{required})}
		\prop{height}[0.35]{bar height}
		\prop{gap}[0.25]{}
		\prop{fill-options}[\{fill=black\}]{TikZ{} option to fill the bar with}
		\prop{draw-options}[\{draw=black!20\}]{TikZ{} option to draw the bar with}
		\prop{label-cs}[identity]{cs name to format labels}
		\prop{*}{all other options are passed to \env{tikzpicture}}
	\end{props}

	\begin{macro}{value}
		Inside \env{barchart}, \cs{value}\marg{label}\marg{percent} is used to add a new bar.
	\end{macro}

\end{Environment}

\subsection{Examples}

\subsubsection{Simple}

\begin{minted}[linenos,fontsize=\footnotesize]{latex}
\begin{barchart}[width=3, draw-options={draw=none}]
  \value{Dim 1}{60}
  \value{Dim 2}{100}
  \value{Dim 3}{70}
  \value{Dim 4}{70}
  \value{Dim 5}{40}
  \value{Dim 6}{60}
\end{barchart}
\end{minted}

\begin{barchart}[width=3, draw-options={draw=none}]
  \value{Dim 1}{60}
  \value{Dim 2}{100}
  \value{Dim 3}{70}
  \value{Dim 4}{70}
  \value{Dim 5}{40}
  \value{Dim 6}{60}
\end{barchart}

\subsubsection{Caped}

Change color, show as a gauge.

\begin{minted}[linenos,fontsize=\footnotesize]{latex}
\begin{barchart}[width=3, label-cs=textinv, fill-options={fill=red!50}, draw-options={draw=red!50}]
  \value{Dim 1}{60}
  \value{Dim 2}{100}
  \value{Dim 3}{70}
  \value{Dim 4}{70}
  \value{Dim 5}{40}
  \value{Dim 6}{60}
\end{barchart}
\end{minted}

\begin{barchart}[width=3, label-cs=textinv, fill-options={fill=red!50}, draw-options={draw=red!50}]
  \value{Dim 1}{60}
  \value{Dim 2}{100}
  \value{Dim 3}{70}
  \value{Dim 4}{70}
  \value{Dim 5}{40}
  \value{Dim 6}{60}
\end{barchart}

\section{bubble chart}

\subsection{Definition}

\begin{Environment}{bubblechart}[\oarg{clist}]
	Environment that hold a bubble chart. \oarg{clist} is a list of the following keyval

	\begin{props}[\oarg{clist}]
		\prop{radius}[1]{max radius}
		\prop{gap}[0.3]{gap between bubbles}
		\prop{fill-options}[\{fill=black\}]{TikZ{} options to fill bubble with}
		\prop{draw-options}[\{draw=black!30\}]{TikZ{} options to draw bubble with}
		\prop{label-cs}[identity]{cs name to format labels}
		\prop{*}{all other options are passed to \env{tikzpicture}}
	\end{props}

	\begin{macro}{value}
		Inside \env{bubblechart}, \cs{value}\marg{label}\marg{percent} is used to add
		a new bubble.
	\end{macro}

\end{Environment}

\subsection{Examples}

\subsubsection{Simple}

\begin{minted}[fontsize=\footnotesize]{latex}
\begin{bubblechart}[draw-options={draw=none}]
	\value{Dim 1}{80}
	\value{Dim 2}{90}
	\value{Dim 3}{50}
	\value{Dim 4}{50}
	\value{Dim 5}{30}
\end{bubblechart}
\end{minted}

\begin{bubblechart}[draw-options={draw=none}]
	\value{Dim 1}{80}
	\value{Dim 2}{90}
	\value{Dim 3}{50}
	\value{Dim 4}{50}
	\value{Dim 5}{30}
\end{bubblechart}

\subsubsection{Caped}

Format labels, change colors, show absolute limit (100\%)

\begin{minted}[fontsize=\footnotesize]{latex}
\begin{bubblechart}[label-cs=textinv, fill-options={fill=red!50}, draw-options={draw=red!50,dashed}]
  \value{Dim 1}{80}
  \value{Dim 2}{90}
  \value{Dim 3}{50}
  \value{Dim 4}{50}
  \value{Dim 5}{30}
\end{bubblechart}
\end{minted}

\begin{bubblechart}[label-cs=textinv, fill-options={fill=red!50},draw-options={draw=red!50,dashed}]
	\value{Dim 1}{80}
	\value{Dim 2}{90}
	\value{Dim 3}{50}
	\value{Dim 4}{50}
	\value{Dim 5}{30}
\end{bubblechart}

\PrintChanges
\PrintIndex

\end{document}
